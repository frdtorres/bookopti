%%%%%%%%%%%%%%%%%%%%%%%%%%%%%%%%%%%%%%%%%%%%%%%%%%%%%%%%%%%%%%%%%%%%%%%%
%  chapters/capitulo1.tex
%%%%%%%%%%%%%%%%%%%%%%%%%%%%%%%%%%%%%%%%%%%%%%%%%%%%%%%%%%%%%%%%%%%%%%%%
\chapter{Mathematical Foundations of Optimization for Data Science}
Autor : \large{Fred Torres Cruz}
\label{chap:1}


\vspace{5em}
% Ejemplo de texto (puedes reemplazar con tu contenido real)
En este capítulo, se presentan los fundamentos matemáticos necesarios 
para entender las técnicas de optimización en ciencia de datos. 
Entre estos fundamentos se incluyen nociones de álgebra lineal, 
cálculo diferencial y análisis de convexidad.

\section{Cálculo y Álgebra Lineal}
% Añade contenido detallado aquí

\section{Convexidad y Optimización}
% Añade contenido detallado aquí

\section{Lagrange Multipliers y Restricciones}
% Añade contenido detallado aquí

% Ejemplo de cita 
% \cite{autor2010}